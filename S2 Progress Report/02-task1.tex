\section{Write a Multifidelity Aeroelastic Optimization Tool}
%\markboth{l}{Write a Multifidelity Aeroelastic Optimization Too}
\label{sec:task1}
\subsection{Description of work}
The code for the implementation of a Multifidelity Aeroelastic Optimization Tool will be written using the OpenMDAO platform and the corresponding Aerostructures Python package developped at ISAE-SUPAERO. The aforementioned platforms have dedicated libraries to solve optimization problems. A GitHub repository will be created to manage the project. The platform allows version control, branching, collaborative work, among other features. 

Potential flow theory will be used as the Low-Fidelity component of the program through its computational implementation: the panel codes (PANAIR). The structural response will be computed using NASTRAN95, this component is not computationally expensive and thus there is no incentive to simplify it any further. The High-Fidelity component will be a CFD solver. 
\subsection{Technical progress}
No significant advances have occurred during this period, as the state of the art review was underway. 
\subsection{Plan vs achievements}
On time for this task.
\subsection{Changes to the original plan}
No significant changes are needed at this point.
\subsection{Planned work for the next months}
Over the next months, the main objective is to get acquainted with the OpenMDAO platform, the aerostructures package and solve a few sample problems and tutorials. Then, once the strategy for the multifidelity integration is established, the development of the first version of the code will begin. 

\section{Implement the Multifidelity Program to a BWB case}
%\markboth{l}{Write a Multifidelity Aeroelastic Optimization Too}
\label{sec:task2}
\subsection{Description of work}
Once the Multifidelity Optimization Program is fully functional, a BWB study case will be analized using the new tool. There are not many reference cases when it comes to BWB configurations. A conceptual design \cite{Quinlan2019} published by NASA  offers useful data for comparison purposes. The multifidelity optimization of a BWB presented by Bryson \cite{Bryson2019a} is another useful source of reference data. Once the model has been at least partially validated with documental sources, the effect of certain parameters on the final optimized BWB can be explored as well. 

At this point, the focus will not only be on the results themselves, but also in the efficiency of the method. It is interesting to compare its performance to other multifidelity method and single fidelity approaches. Refining the algorithm parameters to improve its efficiency and/or accuracy will be possible during the implementation of the study case. 
\subsection{Technical progress}
No significant advances have occurred during this period.
\subsection{Plan vs achievements}
On time for this task.
\subsection{Changes to the original plan}
No significant changes are needed at this point.
\subsection{Planned work for the next months}
For the moment, the development of this task is out of the scope of this progress report.
%%% Local Variables: 
%%% mode: latex
%%% TeX-master: "isae-report-template"
%%% End: 

